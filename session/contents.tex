\title{%
  Session on info theory
}
\author{Daniel Bosk}
\institute{%
  KTH EECS
}

\mode<article>{\maketitle}
\mode<presentation>{%
  \begin{frame}
    \maketitle
  \end{frame}
}

\mode*

\begin{abstract}
  \emph{Summary:}
There are various uses for information theory.
In this session we will explore some of them.

\emph{Intended learning outcomes:}
After this session you should be able
\begin{itemize}
  \item to \emph{understand} how Shannon entropy can be applied in various 
    areas.
\end{itemize}

\emph{Reading:}
You should read on the use of entropy to estimate identifiability,
\citetitle{Eckersley2010hui}~\cite{Eckersley2010hui};
anonymity, \citetitle{MeasuringAnonymity}~\cite{MeasuringAnonymity} and
\citetitle{InfoTheoreticMetricForAnonymity}~\cite{InfoTheoreticMetricForAnonymity} 
(read one of these);
password variety, \citetitle{Komanduri2011opa}~\cite{Komanduri2011opa}.

\end{abstract}


\section{Entropy}

\subsection{Entropy as the expected value}

\begin{frame}
  \begin{block}{Expected value}
    \[
      E[X] = \sum_{x\in X} x \Pr(x).
    \]
  \end{block}

  \pause

  \begin{block}{Shannon entropy}
    \begin{align*}
      H[X]
            &= \sum_{x\in X} \left( \log\frac{1}{\Pr(x)} \right) \Pr(x) \\
            &= -\sum_{x\in X} \log[ \Pr(x) ] \Pr(x).
    \end{align*}
  \end{block}
\end{frame}


\section[Concavity and Jensen]{Concavity and Jensen's inequality}

\subsection{Concave functions}

\begin{frame}[fragile]
  \begin{tikzpicture}
    \begin{axis}[xlabel=$x$,xmin=0.5,ylabel=$\log x$]
      \addplot gnuplot[id=log]{log(x)};
    \end{axis}
  \end{tikzpicture}
\end{frame}

\begin{frame}
  \begin{definition}
    \begin{itemize}
      \item Function \(f\colon \RR\to \RR\) such that
        \begin{align*}
          tf(x) + (1-t)f(y) &\leq f(tx + (1-t)y), \\
          t &\leq 1.
        \end{align*}

      \item Then \(f\) is \emph{concave}.
      \item With strict inequality for \(x\neq y\) we say that \(f\) is 
        \emph{strictly concave}.
    \end{itemize}
  \end{definition}
\end{frame}

\begin{frame}[fragile]
  \begin{example}
    \(\log\colon \RR\to \RR\) is strictly concave.
  \end{example}

  \begin{tikzpicture}
    \begin{axis}[xlabel=$x$,xmin=0.5,ylabel=$\log x$]
      \addplot gnuplot[id=log]{log(x)};
    \end{axis}
  \end{tikzpicture}
\end{frame}

%\begin{frame}
%  \begin{lemma}
%    Låt \(f\) vara en strikt konkav funktion.
%    Då har vi att
%    \begin{align*}
%      tf(x) + (1-y)f(y) = f( tx + (1-t)y )
%    \end{align*}
%    om och endast om \(x = y\).
%  \end{lemma}
%\end{frame}
%
%\begin{frame}
%  \begin{proof}
%    Antag \(x = y\).
%    Då har vi
%    \begin{align*}
%      tf(x) + (1-t)f(x) = f(x)(t+1-t) = f(x).
%    \end{align*}
%    Men
%    \begin{align*}
%      f(tx + (1-t)x) = f((t+1-t)x) = f(x)
%    \end{align*}
%    och alltså har vi likhet.
%
%    Antag \(tf(x) + (1-t)f(y) = f(tx + (1-t)y)\).
%    Då har vi
%    \begin{align*}
%      t( f(x) - f(y) ) + f(y) = f( t( x - y ) + y ).
%    \end{align*}
%    Högerledet implicerar \(f(x) = f(y)\), men då beror vänterledet enbart på 
%    \(f(y)\) och således måste även \(x = y\).
%  \end{proof}
%\end{frame}

\subsection{Jensen's inequality}

\begin{frame}
  \begin{theorem}[Jensen's inequality]
    \begin{itemize}
      \item Concave function \(f\colon \RR\to \RR\).
      \item Real numbers \(a_1, a_2,\ldots, a_n > 0\) such that \(\sum_{i=1}^n 
          a_i = 1\).
      \item Then we have
        \begin{align*}
          \sum_{i=1}^n a_i f(x_i) \leq f\left( \sum_{i=1}^n a_i x_i\right).
        \end{align*}
      \item We have equality iff \(x_1 = x_2 = \cdots = x_n\).
    \end{itemize}
  \end{theorem}
\end{frame}

\begin{frame}
  \begin{block}{Proof of Jensen's inequality}
    \begin{itemize}
      \item By induction.
      \item Assume base case, \(n = 2\).
      \item We have \(a_1 + a_2 = 1 \iff a_1 = 1 - a_2\).
      \item Concavity of \(f\) implies
        \begin{align*}
          a_1f(x_1) + a_2f(x_2) \leq f(a_1x_1 + a_2x_2).
        \end{align*}

      \item Assume true for \(n=k\).
        \begin{align}
          \label{eq:JensenIndhyp}
          \sum_{i=1}^k a_i = 1 \land
          \sum_{i=1}^k a_i f(x_i) \leq f\left( \sum_{i=1}^k a_i x_i\right).
        \end{align}
    \end{itemize}
  \end{block}
\end{frame}

\begin{frame}
  \begin{proof}[Cont.\ proof Jensen's inequality]
    \begin{itemize}
      \item Show also true for \(n = k+1\).
      \item Concavity of \(f\) implies
        \begin{align}
          \label{eq:JensenInductive1}
          \sum_{i=1}^{k+1} a_i f(x_i) &=
            a_1f(x_1) + (1-a_1)\sum_{i=2}^{k+1} \frac{a_i}{1-a_1}f(x_i) \\
          \label{eq:JensenInductive2}
            &\leq
              f\left(a_1x_1 + (1-a_i)\sum_{i=2}^{k+1}
                \frac{a_i}{1-a_1} x_i\right).
        \end{align}
      \item Since \(\sum_{i=2}^{k+1}\frac{a_i}{1-a_i} = 1\) we use the 
        induction hypothesis~\eqref{eq:JensenIndhyp} in eqns~\eqref{eq:JensenInductive1} and~\eqref{eq:JensenInductive2}.
      \item It's thus true for all \(n\in \NN\).
    \end{itemize}
  \end{proof}
\end{frame}


\section[Password guessability]{Password policy and guessability}

\subsection{Password policy lowers entropy}

\begin{frame}
  \begin{example}[Generating an \(n\) character long password]
    \begin{itemize}
      \item Use lower case (26) and upper case (26) characters.
      \item List all possible passwords: \((2\cdot 26)^n\).
      \item Generating is just choosing one uniformly randomly: 
        \(\frac{1}{(2\cdot 26)^n}\).
    \end{itemize}
  \end{example}
\end{frame}

\begin{frame}
  \begin{example}[Generate with a policy]
    \begin{itemize}
      \item Same character set.
      \item Policy: must contain one upper case letter.
      \item Take the previous list, remove all passwords with only lower case 
        letters.
      \item \(\approx (2\cdot 26)^{n-1}\cdot 26\) and \(\frac{2}{(2\cdot 
        26)^n}\)
    \end{itemize}
  \end{example}
\end{frame}

\subsection{Outcomes of humans}

\begin{frame}
  \begin{remark}
    \begin{itemize}
      \item Human secrets are far from uniformly random.
      \item Must sample and estimate randomness.
    \end{itemize}
  \end{remark}
\end{frame}


%%% REFERENCES %%%

\begin{frame}[allowframebreaks]
  \printbibliography
\end{frame}
